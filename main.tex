\documentclass[12pt, letterpaper]{report}
\usepackage[UTF8]{ctex}
\usepackage{plantuml}
\usepackage{tikz}
\usepackage{amsmath}
\usepackage{tikz-uml}
\usepackage{comment}
\usepackage{amssymb}
\usepackage{algorithm}
\usepackage{algpseudocode}
\usepackage{hyperref}
\usetikzlibrary{positioning}
\usetikzlibrary{shapes.multipart}
\usepackage{svg}
\usepackage[pdf]{graphviz}
\usepackage{listings}
\usepackage{fancyvrb}
\lstset
{ %Formatting for code in appendix
    language=C++,
    basicstyle=\ttfamily,
    numbers=left,
    stepnumber=1,
    showstringspaces=false,
    tabsize=1,
    breaklines=true,
    breakatwhitespace=false,
    escapechar=|
}

%\usepackage{algorithmic}
\begin{document}
\title{机密计算抽象}
\author{高炜贺}
\date{2023-12-29}
\maketitle

\newpage

\chapter{可信计算环境的功能和特性}

本文分析基于可信计算环境(TEE)的机密计算(CC),找出众多TEE原型的异同,最终尝试抽象出其共有API集合。面向CC的TEE,通常至少有以下几个特性或功能。

\begin{enumerate}
    \item 内存隔离\\
    内存隔离通过安全硬件,保证数据和指令不被篡改。大多数情况下,我们在思考时,假定TEE在硬件层面是安全的。因此,侧信道攻击(SCA)不在本文讨论范围内。但是,SCA其实是最广泛应用的TEE攻击手段,也是TEE最薄弱的环节。
    \item 可信认证\\
    可信认证通过可信硬件,以及一套可信的认证协议,证明TEE内的代码未被篡改。这一功能可分为本地可信认证(LA)和远程可信认证(RA)。不同应用场景,对二者的需求不同。例如,类似YPC的OLAP通常需要依赖RA;但早期PC上的蓝光4K播放并不依赖实时的在线RA。
    \item 可信密钥\\
    TEE与可信平台模块(TPM)类似,也需要具备至少一个信任根EK。EK通常为设备相关的密钥。TEE设备行为的可信性认证,由该密钥签发。EK还可以生成更多的子密钥,以支持更加灵活的功能。显而易见地,这些子密钥依赖于平台信任根EK。
    \item 可信输入输出\\
    为了提高外设(也包括部分总线设备)的性能,一些TEE(尤其是面向虚拟机的)实例在逻辑上支持与裸金属(bare-metal)虚拟机类似的硬件直通。该功能一般称为可信输入输出(Trusted I/O)。这一功能通常是通过总线加密,或通信线缆加密实现。
\end{enumerate}

一些TEE的实现可能只具备上述功能的一部分,也可能增加更多扩展功能。这都不影响其被认为是TEE。但是,面向OLAP功能的TEE,一般至少要支持类似的功能,以提供安全性和可用性。本文以上述四个功能为基础,首先抽象其对应的API,然后尝试将其组合起来,构成与YPC类似的功能实现。

\chapter{YPC的功能模块}

本章试图模块化地抽象YPC,找出各组件之间的联系和依赖关系。从这些依赖关系中,找出与TEE各功能的对应关系,进而得出YPC对于TEE的依赖情况。

YPC在可信域和不可信域均有重要功能和协议设计。

\chapter{结论}

本文希望将YPC普适性地扩展到众多TEE实现的框架。


\end{document}