\chapter{可信计算环境的功能和特性}

本文分析基于可信计算环境(TEE)的机密计算(CC),找出众多TEE原型的异同,最终尝试抽象出其共有API集合。面向CC的TEE,通常至少有以下几个特性或功能。

\begin{enumerate}
    \item 内存隔离\\
    内存隔离通过安全硬件,保证数据和指令不被篡改。大多数情况下,我们在思考时,假定TEE在硬件层面是安全的。因此,侧信道攻击(SCA)不在本文讨论范围内。但是,SCA其实是最广泛应用的TEE攻击手段,也是TEE最薄弱的环节。

    \begin{lstlisting}
// untrusted memory ops 
bytes u_mem_rd(src_addr, length); 
u_mem_wr(dest_addr, bytes, length); 
        
// trusted memory ops
bytes t_mem_rd(src_addr, length);
t_mem_wr(dest_addr, bytes, length);         
    \end{lstlisting}

    \item 可信认证\\
    可信认证(attestation)通过可信硬件,以及一套安全的认证协议,证明TEE内的数据和代码未被篡改。这一功能可分为本地可信认证(LA)和远程可信认证(RA)。不同应用场景,对二者的需求不同。例如,类似YPC的OLAP框架通常至少需要依赖RA;但早期PC上的蓝光4K播放并不依赖实时的在线RA。

    \begin{lstlisting}
// attestations 
Report local_attest(
    TargetInfo target_info
) {
    // 1. generate report within target enclave
    // 2. send the report to attesting enclave
    // 3. retrieve report from attesting enclave and return 
}; 
Report remote_attest(
    TargetInfo target_info
) {
    // 1. generate report in target enclave
    // 2. send report to QE
    // 3. retrieve RA report and return
};   
    \end{lstlisting}

    \item 可信密钥\\
    TEE与可信平台模块(TPM)类似,也需要具备至少一个信任根EK。EK通常为设备相关的密钥。TEE设备行为的可信性认证,由该密钥签发。EK还可以生成更多的子密钥,以支持更加灵活的功能。显而易见地,这些子密钥依赖于平台信任根EK。

    \begin{lstlisting}
// Endorsement key 
Certificate get_ek_certificate(); 
    \end{lstlisting}

    \item 可信输入输出\\
    为了提高外设(也包括部分总线设备)在TEE场景下的性能,一些(尤其是面向虚拟机的)TEE设计方案在逻辑上支持与裸金属(bare-metal)虚拟机类似的硬件直通。该功能一般称为可信输入输出(Trusted I/O)。它通常是通过计算机总线加密,或通信线缆加密实现。
\end{enumerate}

一些TEE的实现可能只具备上述功能的一部分,也有一些产品支持更多扩展功能。这些差异,都不影响其被认为是TEE。但是,面向OLAP功能的TEE,一般至少要支持类似的功能,以提供特定场景下的安全性和可用性。本文以上述四个功能为基础,首先抽象其对应的API,然后尝试将其组合起来,构成与YPC类似的功能实现。